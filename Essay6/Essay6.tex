\documentclass[12pt]{article}
\usepackage[margin=1in]{geometry}
\usepackage[doublespacing]{setspace}
\usepackage{indentfirst}
\newif{\ifdraft}
\draftfalse

\begin{document}
Anthony Le

Prof. Roger Hickman

MUS468 (Section 2)

\today

\begin{center}
    \large Alternatives to Orchestal Score Style
\end{center}

\ifdraft{}
\section*{\underline{Prompt:}}
\noindent Between 1977 and 1984 there were numerous alternatives to the full orchestral score style. 
Describe these alternatives and give examples mentioned in the text from this period. 
Films of John Williams, the foremost exponent of orchestral scores, should NOT be considered as an alternative. 

\section*{\underline{Brainstorm:}}
\begin{itemize}
    \item Alternative scores:
    \begin{itemize}
        \item Synthesized scores
        \item Popular Music
        \item Adapted Scores
    \end{itemize}
    \item \underline{Synthesized Scores:}
    \begin{description}
        \item[Description:]
        \item[] 
        \begin{itemize}
            \item Synthesized scores are made from a synthesizer, an electronic instrument that is able to 
            generate and reproduce sounds. Since they can reproduce sounds, this instrument can replace an
            entire orchestra, or make new sounds that traditional instruments can not create.
            \item There are three styles of synthesizer music that can be heard in film music.
            \item Modern synthesized songs are associated with electronic music and the instrument can be 
            used to create new scores that have new colors.
            \item Traditional synthesized songs utilize the instrument's ability to replicate sounds in order 
            to imitate acoustic instruments or instruments within an orchestra (piano, violin, flute, etc.).
            \item Popular synthesized songs are associated with rock music and are an important element of popular-music
            film scores.
        \end{itemize}
        \item[Notable Films:]
        \item[] 
        \begin{itemize}
            \item Some notable films that utilize synthesized music in its score include \emph{Midnight Express},
            \emph{Halloween}, and \emph{Vangelis}.
            \item In \emph{Midnight Express}, composer Moroder utilized the synthesizer in a tradiional sense from dissonances
            to popular melodic materials. He mimicked sounds of musical instruments like the piano, violin, and electric guitars
            while composing his score for the film.
            \item In \emph{Halloween}, the synthesizer is used in a modern sense creating the theme for Michael with uneven, lopsided 
            rhythm done in a low register reflects the killer's unrelenting pursuit and murder of his victims. 
        \end{itemize}  
    \end{description}
    \item \underline{Popular Music:}
    \begin{description}
        \item[Description:]
        \item[] 
        \begin{itemize}
            \item Popular music was already been well established in all phases of film scoring.
            \item From the late 1970s and into the 1980s there is rapid trend changes and divisions in 
            the alternative score breaking into distinct styles and sub-styles that gave a variety of new fresh sounds.
        \end{itemize}
        \item[Notable Films:]
        \item[] 
        \begin{itemize}
            \item Some notable films that uses this alternative scoring include \emph{Rocky}, \emph{Taxi Driver}, \emph{Nashville},
            and \emph{Saturday Night Fever}.
            \item In \emph{Saturday Night Fever}, the film's plot is designed with numerous scenes that gave opportunities for the director
            to play popular music. The film brought the disco genre into mainstream audiences, and the soundtrack by Bee Gees that sold with the film surpassed all
            sale records of the time for film music.
            \item \emph{Rocky} was a sports film that featured traning sessions and competitions. There is a heavy use of a combo of
            popular rock music and ochestral instruments such as brass fanfare to incite strong emotions in the main character's journey in boxing.
        \end{itemize} 
    \end{description}
    \item \underline{Adapted Scores:}
    \begin{description}
        \item[Description:]
        \item[] 
        \begin{itemize}
            \item Adapted scores from the late 1970s and early 1980s adapted numerous exerpts from 18th century works.
            Specifically baroque era music acheived popularity since it was a distinctively fresh sound that is dominant
            in Romantic, moden, and popular styles.
        \end{itemize}
        \item[Notable Films:]
        \item[] 
        \begin{itemize}
            \item Some notable films that use adapted scores are \emph{The Shining} and \emph{Amadeus}.
            \item In \emph{The Shining}, uses borrowed excerpts of original works from the twentieth centure such as
            Béla Barték’s Music for Strings, Percussion,and Celesta (1936), Gydray Ligeti’s Lontano (1967), and six works by the 
            Polish avant-garde composer Kzysztof Penderecki in combination with Wendy Carlos original synthesized music.
            \item For \emph{Amadeus}, the film uses no newly composed music, but rather music by Mozart. 
        \end{itemize}  
    \end{description}
\end{itemize}
\fi

\ifdraft{}
\section*{\underline{Paragraph 1: Introduction}}
\fi
From 1977 to 1984, there were numerous alternative fim scores to the full orchestral score style
that John Williams with his \emph{Star War} films utilized. Many composers during that time felt
that there is no need to use symphonic underscoring to create a high-quality film. 

\ifdraft{}
\section*{\underline{Paragraph 2: Synthesized Scores}}
\fi
Synthesized scores are made from a synthesizer, an electronic instrument that is able to 
generate and reproduce sounds. Since they can reproduce sounds, this instrument can replace an
entire orchestra, or make new sounds that traditional instruments can not create.
There are three styles of synthesizer music that can be heard in film music.
Modern synthesized songs are associated with electronic music and the instrument can be 
used to create new scores that have new colors.
Traditional synthesized songs utilize the instrument's ability to replicate sounds in order 
to imitate acoustic instruments or instruments within an orchestra (piano, violin, flute, etc.).
Popular synthesized songs are associated with rock music and are an important element of popular-music
film scores.
Some notable films that utilize synthesized music in its score include \emph{Midnight Express},
\emph{Halloween}, and \emph{Vangelis}.
In \emph{Midnight Express}, composer Moroder utilized the synthesizer in a tradiional sense from dissonances
to popular melodic materials. He mimicked sounds of musical instruments like the piano, violin, and electric guitars
while composing his score for the film.
In \emph{Halloween}, the synthesizer is used in a modern sense creating the theme for Michael with uneven, lopsided 
rhythm done in a low register reflects the killer's unrelenting pursuit and murder of his victims. 



\ifdraft{}
\section*{\underline{Paragraph 3: Popular Music}}
\fi
Popular music was already been well established in all phases of film scoring.
From the late 1970s and into the 1980s there is rapid trend changes and divisions in 
the alternative score breaking into distinct styles and sub-styles that gave a variety of new fresh sounds.
Some notable films that uses this alternative scoring include \emph{Rocky}, \emph{Taxi Driver}, \emph{Nashville},
and \emph{Saturday Night Fever}. In \emph{Saturday Night Fever}, the film's plot is designed with numerous scenes 
that gave opportunities for the director to play popular music. The film brought the disco genre into mainstream audiences, 
and the soundtrack by Bee Gees that sold with the film surpassed all sale records of the time for film music.
\emph{Rocky} was a sports film that featured traning sessions and competitions. There is a heavy use of a combo of
popular rock music and ochestral instruments such as brass fanfare to incite strong emotions in the main character's journey in boxing.



\ifdraft{}
\section*{\underline{Paragraph 4: Adapted Scores}}
\fi
Adapted scores from the late 1970s and early 1980s adapted numerous exerpts from 18th century works.
Specifically baroque era music acheived popularity since it was a distinctively fresh sound that is dominant
in Romantic, moden, and popular styles.
Some notable films that use adapted scores are \emph{The Shining} and \emph{Amadeus}.
In \emph{The Shining}, uses borrowed excerpts of original works from the twentieth centure such as
Béla Barték’s Music for Strings, Percussion,and Celesta (1936), Gydray Ligeti’s Lontano (1967), and six works by the 
Polish avant-garde composer Kzysztof Penderecki in combination with Wendy Carlos original synthesized music.
For \emph{Amadeus}, the film uses no newly composed music, but rather music by Mozart. 


\end{document}