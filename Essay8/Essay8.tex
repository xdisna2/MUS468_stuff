\documentclass[12pt]{article}
\usepackage[margin=1in]{geometry}
\usepackage[doublespacing]{setspace}
\usepackage{indentfirst}
\newif{\ifdraft}
\draftfalse

\begin{document}
Anthony Le

Prof. Roger Hickman

MUS468 (Section 2)

\today

\begin{center}
    \large Minimalism's Impact
\end{center}

\ifdraft{}
\section*{\underline{Prompt:}}
\noindent Describe the impact of musical minimalism on film music in the 21st century. 
Be sure to include the concept of action minimalism and to cite specific films and composers.

\section*{\underline{Brainstorm:}}
\begin{itemize}
    \item \underline{Three Films:}
    \begin{itemize}
        \item Avatar
        \item Inception
        \item Interstellar
    \end{itemize}
    \item Music from \emph{The Lord of the Rings} reveals a transition between old and new
    approaches to film scoring. Famous films from this era have scores that are dominated by melodies, and have leimotifs that
    are memorable and plants an impression into the viewer's imagination on the story itself. However, minimalistic film scoring
    has made its way with Shore including a number of musical passage of minimalistic structure such as having a single pitch or chord.
    \item Minimalistic approaches to film scoring emphasizes rhythmic drives and the avoidance of tuneful melodies in these following films.
    \item \underline{Avatar:}
    \begin{description}
        \item[Plot]
        \item[] 
        \begin{itemize}
            \item Directed by James Cameron, the one who also directed \emph{Titanic}, \emph{Avatar} (2009) was a big 
            box office winner. James Horner, the composer, composes a score that provides an aural link between the Na'vi and 
            non-Western populations. 
        \end{itemize}
        \item[Minimalism]
        \item[]   
        \begin{itemize}
            \item For example, he utilizes a Japanese wooden flute, a shakuhachi, for scenes prevelant 
            in regards to visions of the Pandorian forest. Nasal ornamental vocal melodies, choral chanting, and lots of drumming
            were some of the sounds in the film that associates with the Na'Vi civilization. 
            \item Minimalism was prevalent in this film by composing a score that lacks a defining melody in comparison to his 
            previous work in \emph{Braveheart} (1995). A distinctive piece in Avatar tha reflects this is the Avatar/Love Theme which 
            is a simple four-note rising idea that can be heard numerous times in visions of Pandora. Having repeating patterns and sustained sounds
            such as the love theme is credible of minimalism having an effect on the score of the film.
            \item Simple folklike melodies are sung by a choir of children which are assisted further with orchestral instruments such as 
            the French horn and strings. There is very little need for lyrics or complex melodies to say to the audiene that the location is 
            exotic, pure, and reflect the people's innocence that inhabit the planet.
        \end{itemize}
    \end{description}
    \item \underline{Inception:}
    \begin{description}
        \item[Plot]
        \item[] 
        \begin{itemize}
            \item Christopher Nolan's \emph{Inception} (2010) is an imaginative fantasy that depicts a world with technological 
            advancements to allow people to share dreams, and explores the concept of the dream world. Cobb, an infiltrator of dreams to extract corporate
            secrets from the rich and powerful, develops a technique that allows him to create multilayered dreams.
        \end{itemize}
        \item[Minimalism]
        \item[]   
        \begin{itemize}
            \item The influence of minmalistic scoring of films can be found in Han Zimmer's work for \emph{Inception}. By having sustained sounds and
            repeating patterns, Zimmer is able to support the vision of dream worlds without too much work in emphasizing the dream state. 
            \item For example, Zimmer incorporates music into three aspects of the story: action on dream levels, singals of a dream ending, and a limbo state
            where a dream has yet to be constructed. Within each dream the pace of action increases, hence the use of percussive instruments playing at different 
            layers and paces generate an emotion of climax and excitement.
            \item Everytime a dream ends, a song is played rather than a consistent melody that is associated with the end of dreams. 
            The song chosen is accompanied by a piano only deriving its repeating rhythm and half-step motion of the bass line. 
            \item The film ends and opens in limbo where cues of simple, retative sounds are played such as the wind. 
        \end{itemize}
    \end{description}
    \item \underline{Interstellar:}
    \begin{description}
        \item[Plot]
        \item[] 
        \begin{itemize}
            \item Another collaborative film by Christopher Nolan and Hans Zimmer, \emph{Intersellar} (2014) is a sci-fi fil that 
            follows astronaut Cooper on a hopeless mission to find another livable planet near Saturn. With all of his struggles, Cooper
            is able to find a planet through paternal love between father and daughter. 
        \end{itemize}
        \item[Minimalism]
        \item[]   
        \begin{itemize}
            \item Influences of minimalism can be heard through the usage of repetition and long sustained chords paired with electronic and
            an organ. Furthermore, the most effective example would be the simple sound of a clock ticking which generates great tension in times of
            essence. 
            \item Simple long chords of an organ chord can be heard in the 2001 theme with a beginning low register of the organ pedal concluding with
            an isolated sound of an organ chord. 
        \end{itemize}
    \end{description}
\end{itemize}
\fi

\ifdraft{}
\section*{\underline{Paragraph 1: Introduction}}
\fi
Music from \emph{The Lord of the Rings} reveals a transition between old and new
approaches to film scoring. Famous films from this era have scores that are dominated by melodies, and have leimotifs that
are memorable and plants an impression into the viewer's imagination on the story itself. However, minimalistic film scoring
has made its way with Shore including a number of musical passage of minimalistic structure such as having a single pitch or chord.
Minimalistic approaches to film scoring emphasizes rhythmic drives and the avoidance of tuneful melodies in these following films.

\ifdraft{}
\section*{\underline{Paragraph 2: Avatar}}
\fi
Directed by James Cameron, the one who also directed \emph{Titanic}, \emph{Avatar} (2009) was a big 
box office winner. James Horner, the composer, composes a score that provides an aural link between the Na'vi and 
non-Western populations. 
For example, he utilizes a Japanese wooden flute, a shakuhachi, for scenes prevelant 
in regards to visions of the Pandorian forest. Nasal ornamental vocal melodies, choral chanting, and lots of drumming
were some of the sounds in the film that associates with the Na'Vi civilization. 
Minimalism was prevalent in this film by composing a score that lacks a defining melody in comparison to his 
previous work in \emph{Braveheart} (1995). A distinctive piece in Avatar tha reflects this is the Avatar/Love Theme which 
is a simple four-note rising idea that can be heard numerous times in visions of Pandora. Having repeating patterns and sustained sounds
such as the love theme is credible of minimalism having an effect on the score of the film.
Simple folklike melodies are sung by a choir of children which are assisted further with orchestral instruments such as 
the French horn and strings. There is very little need for lyrics or complex melodies to say to the audiene that the location is 
exotic, pure, and reflect the people's innocence that inhabit the planet.

\ifdraft{}
\section*{\underline{Paragraph 3: Inception}}
\fi
Christopher Nolan's \emph{Inception} (2010) is an imaginative fantasy that depicts a world with technological 
advancements to allow people to share dreams, and explores the concept of the dream world. Cobb, an infiltrator of dreams to extract corporate
secrets from the rich and powerful, develops a technique that allows him to create multilayered dreams.
The influence of minmalistic scoring of films can be found in Han Zimmer's work for \emph{Inception}. By having sustained sounds and
repeating patterns, Zimmer is able to support the vision of dream worlds without too much work in emphasizing the dream state. 
For example, Zimmer incorporates music into three aspects of the story: action on dream levels, singals of a dream ending, and a limbo state
where a dream has yet to be constructed. Within each dream the pace of action increases, hence the use of percussive instruments playing at different 
layers and paces generate an emotion of climax and excitement.
Everytime a dream ends, a song is played rather than a consistent melody that is associated with the end of dreams. 
The song chosen is accompanied by a piano only deriving its repeating rhythm and half-step motion of the bass line. 
The film ends and opens in limbo where cues of simple, retative sounds are played such as the wind. 

\ifdraft{}
\section*{\underline{Paragraph 4: Interstellar}}
\fi
Another collaborative film by Christopher Nolan and Hans Zimmer, \emph{Intersellar} (2014) is a sci-fi fil that 
follows astronaut Cooper on a hopeless mission to find another livable planet near Saturn. With all of his struggles, Cooper
is able to find a planet through paternal love between father and daughter. 
Influences of minimalism can be heard through the usage of repetition and long sustained chords paired with electronic and
an organ. Furthermore, the most effective example would be the simple sound of a clock ticking which generates great tension in times of
essence. 
Simple long chords of an organ chord can be heard in the 2001 theme with a beginning low register of the organ pedal concluding with
an isolated sound of an organ chord. 

\end{document}