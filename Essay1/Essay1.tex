\documentclass[12pt]{article}
\usepackage[margin=1in]{geometry}
\usepackage[doublespacing]{setspace}
\usepackage{indentfirst}
\newif{\ifdraft}
\draftfalse
\begin{document}
Anthony Le

Prof. Roger Hickman

MUS468 (Section 2)

\today

\begin{center}
    \large Film Music Forerunners
\end{center}

\ifdraft{}
\section*{\underline{Prompt}}
\noindent Chapter 4 discusses a variety of 19th-century musical types that can be considered to be forerunners of film music. Describe three of these, including their basic qualities, important composers, and similarities to film music. At least one should be a theatrical forerunner.

\section*{\underline{Brainstorm}}
\begin{itemize}
    \item Three forerunners of film music (one has to be theatrical forerunner)
    \begin{itemize}
        \item Basic qualities
        \item Important composers
        \item Similarities to film music
    \end{itemize}
    \item What musical type(s) is forerunners of film music?
    \begin{itemize}
        \item Instrumental (Program Music)
        \item Opera
        \item Ballet
    \end{itemize}

    \item \textbf{Program Music}
    \begin{itemize}
        \item \underline{Basic Qualities:}
        \begin{itemize}
            \item Mimics natural sounds
            \item Suggests emotions and produces images 
            \item Can be broken into many movements or acts 
        \end{itemize}
        \item \underline{Important Composers:}
        \begin{itemize}
            \item Beethoven (Symphony No. 6)
            \item Hector Berlioz (Symphony fantastique)
        \end{itemize}
        \item \underline{Similarities to Film Music:}
        \begin{itemize}
            \item Just like films can be broken into many acts or scenes, program music can be in many movements which are similar
        \end{itemize}
    \end{itemize}

    \item \textbf{Ballet}
    \begin{itemize}
        \item \underline{Basic Qualities:}
        \begin{itemize}
            \item No spoken dialogue
            \item Storytelling is done through dance in correlation to music
            \item Supports a range of emotions, actions, and moods 
        \end{itemize}
        \item \underline{Important Composers:}
        \begin{itemize}
            \item Pyotr Tchaikovsky (Sleeping Beauty, Swan Lake, The Nutcracker)
            \item Adolphe Adam 
        \end{itemize}
        \item \underline{Similarities to Film Music:}
        \begin{itemize}
            \item Like Le'motifs, in ballet music themes can denote a character's appearance
            \item Conveys emotion, mood, and actions 
        \end{itemize}
    \end{itemize}

    \item \textbf{Opera}
    \begin{itemize}
        \item \underline{Basic Qualities:}
        \begin{itemize}
            \item theatrical medium based on singing, perdominantly written for instruments
            \item utilizes music to be able to express a full range of emotions, using the voice as a medium
            \item Performed in a modified theater
        \end{itemize}
        \item \underline{Important Composers:}
        \begin{itemize}
            \item Richard Wagner (Tristan und Isolde, Der Ring des Nibelungen)
            \item Georges Bizet (Carmen)
        \end{itemize}
        \item \underline{Similarities to Film Music:}
        \begin{itemize}
            \item Conveys a wide range of emotions
            \item Listened in a theater as watching a film 
        \end{itemize}
    \end{itemize}
\end{itemize}
\fi

%\section*{\underline{Paragraph 1: Introduction}}
There were a variety of 19th-century musical types that are considered to be forerunners of film music. 
Three of these forerunners are instrumental program music, opera, and ballet. Each of these musical types are
distinguished by their basic qualities, relevant composers, and their similarities to film music. Some of the works
within these musical types are written by pretty well known composers that many know and have heard today.

%\section*{\underline{Paragraph 2: Program Music}}
The first of these forerunners is program music. Program music can suggest a wide range of emotions from joy to anger. 
Furthermore, program music has the ability by itself to tell a story by suggesting an image through its instrumental nature. For example,
in Beethoven's symphonies he adds in some imagery to the viewer's ears by mimicing certain sounds found in nature like thunder through the 
timpani drums, or bird sounds recreated through woodwind instruments. Also program music like how films are broken down into scenes can be 
split into many parts as well called movements. Program music or symphonies are normally four or more movements in relation to the multiple
acts within films. Both Beethoven's Symphony No. 6 (\emph{Pastoral}) and Berlioz's \emph{Symphony fantastique} are program symphonies that have five movements. As mentioned
frequently, some noticible composers of program music are Beethoven and Berlioz. Program music share some similarities to film music in such ways 
that it inflicts emotion and creates a sense of imagery through its mimicing of natural sounds setting the scene and its mood. Furthermore, like the films 
to which the music is played within, program music has numerous parts or momvements just like how there are many acts or scenes in a movie film.

%\section*{\underline{Paragraph 3: Ballet Music}}
Coming up from program music is ballet music. Similar to program music, ballet has some unique qualities that breaks away from the instrumental 
aspect of program music. Ballet is one of the three music types from the 19th century that is a theatrical medium. In ballet, stories are told in 
rhythmic dance and the characters actions. There is no dialogue spoken at all. Through the dancers actions and dance, a wide range of emotions and moods is visually 
shown to the audience while the music invokes their imagination on the depth of that emotion. Some noteable composers of ballet music include Adolphe Adam, and Pyotr
Tchaikovsky. Tchaikovsky was more known with his works such as \emph{Sleeping Beauty}, \emph{Swan Lake}, and \emph{The Nutcracker}.
Some similarities ballet music has to film music such as in ballet there are music themes that can denote a character's appearance in a scene similar to Le'motifs in film music.
For example, in Tchaikovsky's \emph{Swan Lake}, the themes for both the White and Black Swan are the same, however their melodies are played with different instruments to represent
the contrasting good against evil. Finally, like film music, ballet can also convey a lot of emotion and moods.

%\section*{\underline{Paragraph 4: Opera Music}}
The last of these forerunners of music types in the 19th century is Opera. Like ballet, opera music is another theatrical medium that unlike ballet which utilizes dancing and actions
is based on singing. Although, unlike ballet where the music is written dependent on the dance, opera music is quite the opposite where the music is written mainly for the instruments not the singers.
Music is utilized to express a full range of emotions while using the voice and the characters on stage as a medium to further enhance these aspects. Furthermore, opera itself was performed in a 
modified theater that Wagner designed to fully implement his control and ideals over his productions. By having a darkened auditorium for performance, detailed scenery, visual effects per scene, and many
more theatrical innovations by him, this was the foundations to the modern theater seating for future films. Some important composers of opera music include Richard Wagner, and Georges Bizet. Wagner was 
a more prominant composer for opera music with his works such as \emph{Tristan und Isolde}, and the Ring Cycle (\emph{Der Ring des Nibelungen}). Similarities that opera music shares with film music is obviously
the setting at which it is played, in a dark theater to enhance the sensory appeal of the audience. Furthermore, like film music a wide range of emotions are conveyed to the audience.


\end{document}