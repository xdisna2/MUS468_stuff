\documentclass[12pt]{article}
\usepackage[margin=1in]{geometry}
\usepackage[doublespacing]{setspace}
\usepackage{indentfirst}
\newif{\ifdraft}
\draftfalse

\begin{document}
Anthony Le

Prof. Roger Hickman

MUS468 (Section 2)

\today

\begin{center}
    \large Classical Film Genres and their Music
\end{center}

\ifdraft{}
\section*{\underline{Prompt:}}
\noindent The Golden Age of Hollywood (1929-1943) created film genres defined by a variety of conventions and clichés. 
Cite three genres from this period and describe the typical music found in those genres. 
Be sure to include specific examples from films mentioned in the text.

\section*{\underline{Brainstorm:}}
\begin{itemize}
    \item \underline{Three Genres of Films:}
    \begin{itemize}
        \item Horror
        \item Romance
        \item Epics
    \end{itemize}
    \item \underline{Horror:}
    \begin{description} 
        \item[Types of Music Used:]
        \item[] 
        \begin{itemize}
            \item \underline{King Kong:}
            \begin{itemize}
                \item Utilized raucous brass and pounding percussion 
                \item Harsh harmonies, brass flutter-tonguing, and stuttering of repeated chords 
                gave a horrifying sound as cliché as the genre itself which is meant to incite fear
                into the audience.
                \item The music goes beyond its purpose of the horrifying and scary nature of Kong by also representing
                the human personality that the creature has.
                \item Usage of leitmotifs and thematic transformations were used to present two principal themes: Beauty
                and the Beast to which both respective characters slowly change as the film goes on.  
                \item Usage of music in the film was spread out and its pace was unrelenting from scene to scene.
                For example, in the beginning no music is used for twenty minutes, then quickly the music starts
                when a ship encounters a fog, and becomes unrelenting at the end during the battle between King Kong
                and the T-rex.
            \end{itemize}
            \item \underline{The Bride of Frankenstein:}
            \begin{itemize}
                \item Waxman the composer reflects the concept of \emph{self-parody}, treating the conventions of 
                a genre with humor, by having a film score that has its own sense of parody. For example, the leitmotifs
                of Frankenstein and his bride are exaggerated verions of their horrifying characters.
                \item Dissonance, a "wrong-note" harmony that is created by the accidental playing of two adjacent pitches,
                is used in The Monster's motive, which in turns create a jarring and horrific effect.
                \item In retrospect, the bride of Frankenstein has a theme that is lyrical and heavenly to depict her beauty in 
                Frankenstein's eyes.
            \end{itemize}
        \end{itemize} 
        \item[Notable Films:] \emph{King Kong}, \emph{Bride of Frankenstein}  
    \end{description}
    \item \underline{Romance:}
    \begin{description}
        \item[Description of Genre:] Subcategory of the drama genre, it is an independent type that can conquer other
        film genres such as Horror and Epics. 
        \item[Types of Music Used:]
        \begin{itemize}
            \item []
            \item \underline{Wuthering Heights:}
            \begin{itemize}
                \item Generally has a full orchestral score like the previous two genres, a perdominant string timbre, and a beautiful
                central theme. 
                \item Utilization of strings, the thee between the love of Heathcliff and Cathy represents the
                warmth and endurance of their love.
                \item The score that Newman has composed does not follow the traditional role that film music sets when it comes to
                its representation in the changes in emotion or mood of the scene. 
                \item Overall the love theme overcomes all emotions, regardless of Cathy's death to signify joy and ultimate consummation of their
                strong love for each other.
            \end{itemize}
        \end{itemize} 
        \item[Notable Films:] \emph{Wuthering Heights}
    \end{description}
    \item \underline{Epics:}
    \begin{description}
        \item[Description of Genre:] Historical settings and extended stories of a person of an era. 
        \item[Types of Music Used:] 
        \item[] 
        \begin{itemize}
            \item \underline{Gone with the Wind}
            \begin{itemize}
                \item Variety of instrumental and vocal musics such as dance orchestras, organ music, band music, and choir chorus.
                \item Borrowings of short quotations from known songs such as "Dixie", "Old Folks at Home", and "Maryland, My Maryland".
                \item leitmotifs of characters reflect their ethic backgrounds such as jig underscores for Gerald O'Hara's Irish heritage or
                synocopated rhythmic beats with African Americans.
                \item Steiner also represents some of the character's traits such as Melanie's serene melody or Butler's confident/strong theme.
            \end{itemize}
        \end{itemize}
        \item[Notable Films:] \emph{Gone with the Wind}
    \end{description}
\end{itemize}
\fi

\ifdraft{}
\section*{\underline{Paragraph 1: Introduction}}
\fi
During the Golden Age of Hollywood, there arose many different film genres that each had their own
conventions and clichés. Some of these genres include Horror, Romance, and Epics which to each genre 
gave rise to some films that had music which best represents the clichés of each genre.


\ifdraft{}
\section*{\underline{Paragraph 2: Horror}}
\fi
The horror film genre had two notable films: \emph{King King}, and \emph{The Bride of Frankenstein}.
\emph{King Kong}, musical score composed by Max Steiner, utilized raucous brass and pounding percussion in many of
the scenes of the film. With harsh harmonies, brass flutter-tonguing, and stuttering of repeating chords, this gave a 
horrendous sound that as cliché as the genre itself intends to incite fear into the audience of King Kong, and the events 
that arose within the film. Furthermore, the music goes above and beyond by taking on a thematic transformation of the characters
representing Beauty and the Beast. For example, in the beginning King Kong's theme depicted him as a terrifying creature, however
as the film goes on his theme then takes on a humanistic representation of the creature. In \emph{The Bride of Frankenstein}, composed
by Waxman, he reflects on the concept of \emph{self-parody}, treating the conventions of a genre with humor, within the film's music by 
having a film score that has its own sense of comedy. The film furthermore utilizes dissonance, a "wrong-note" harmony that is accidentally created by 
playing two adjacent pitches, in The Monster's motive to create a jarring and horrific effect of Frankenstein.

\ifdraft{}
\section*{\underline{Paragraph 3: Romance}}
\fi
The next genre is romance which is considered as a subcategory of the drama genre. Romance is also 
considered as independent since it has the ability to coexist with other genres of film such as horror and epics.
A notable film in the romance genre is \emph{Wuthering Heights}. Within the romance genre, the score for these films
generally has a full orchestral score like the previous two genres, a perdominant string timbre, and a beautiful
central theme. \emph{Wuthering heights}, composed by Newman, utilizes the string family of instruments to represent 
the warmth and endurance of Heathcliff and Cathy love for each other. The score that Newman has composed does not follow 
the traditional role that film music sets when it comes to its representation in the changes in emotion or mood of the scene.
Overall the love theme overcomes all emotions, regardless of Cathy's death to signify joy and ultimate consummation of their
strong love for each other.

\ifdraft{}
\section*{\underline{Paragraph 4: Epics}}
\fi

The final genre is epics which is generally filmed in a historical setting and has an extended story about a person coming from
that era. A notable film in the epic genre is \emph{Gone with the Wind}. The score in the film uses a variety of instrumental and 
vocal musics such as dance orchestras, organ music, band music, and choir chorus. There are many borrowings of short quotations from 
known songs such as "Dixie", "Old Folks at Home", and "Maryland, My Maryland". The leitmotifs of the characters ethnic backgrounds are 
also reflective of their melodies. For example, jig underscores are used for Gerald O'Hara's Irish heritage or synocopated rhythmic 
beats with African Americans.
 
\end{document}