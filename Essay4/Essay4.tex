\documentclass[12pt]{article}
\usepackage[margin=1in]{geometry}
\usepackage[doublespacing]{setspace}
\usepackage{indentfirst}
\newif{\ifdraft}
\draftfalse

\begin{document}
Anthony Le

Prof. Roger Hickman

MUS468 (Section 2)

\today

\begin{center}
    \large Popular Music in Film
\end{center}

\ifdraft{}
\section*{\underline{Prompt:}}
\noindent Describe the expanding role of popular music (jazz, country, and rock) in films from during the 1950s. 
Be sure to mention films from the textbook.

\section*{\underline{Brainstorm:}}
\begin{itemize}
    \item \underline{Popular Music Types:}
    \begin{itemize}
        \item jazz
        \item country
        \item rock
    \end{itemize}
    \item \underline{Jazz:}
    \begin{description}
        \item[Facts:]
        \item[] 
        \begin{itemize}
            \item Born from film noir, a dark and pessimistic film style, the sound of jazz
            featured a music type that utilizes cutting and coarse melodic line that is created 
            generally from a saxophone or trumpet
            \item Jazz was mainly employed to suggests seedier aspects of life such as sex and violence.
            \item Futhermore, jazz was also employed for instrumental cues due to its cutting and coarse melodic line.
        \end{itemize}
        \item[Notable Films with Songs:]
        \item[] 
        \begin{itemize}
            \item \emph{A Streetcar Named Desire}
            \begin{itemize}
                \item Set in New Orleans which was the birthplace of jazz music, Alex North positions jazz source music of the film
                is naturally placed in all settings like barrooms and restaurants which entails an atmosphere 
                of lust, instability, and passion. 
                \item The underscoring nature of jazz was utilized in suggesting a sexually charged atmosphere of Stanley's House.
                \item There is a slow transition throughout the film from popular music to instrumental cues such as the rape of his 
                sister-in-law.
            \end{itemize}
            \item \emph{Touch of Evil}
            \begin{itemize}
                \item A film that has two versions, Welles's \emph{Touch of Evil} (1958) was the last significant 
                film noir that was surrounded by controversy on its production.
                \item The film incorporates many popular music styles in many of its scenes. Loud and intense jazz music is utilized
                during Hank Quinlan's murder of Uncle Joe Grande, which associates violence.
            \end{itemize}
            \item \emph{Anatomy of a Murder}
            \begin{itemize}
                \item Directed by Otto Preminger, he hired Duke Ellington with the intent of incorporating jazz into the film.
                \item Jazz was mainly used during the contribution of discussions in regards to rape, semen, and "panties".
            \end{itemize}
        \end{itemize}  
    \end{description}
    \item \underline{Country:}
    \begin{description}
        \item[Facts:]
        \item[] 
        \begin{itemize}
            \item Country or western songs was another popular music type. 
            \item Its usage in films was pioneered by Dimitri Tiomkin, a veteran Hollywood composer, that gave rise to theme songs.
            \item Theme songs became quickly popular as it served as a new source of income for recordings, and served as 
            free publicity for the films that had the same name as the songs.
        \end{itemize}
        \item[Notable Films with Songs:]
        \item[] 
        \begin{itemize}
            \item \emph{High Noon}
            \begin{itemize}
                \item A classic film featuring a showdown between good and evil. 
                \item The film lacked features of a true western of the age, the 
                true essence of the film is not the showdown, but the plot tensions that
                leads to the climax.
                \item The film is more known for its scores such as the theme song \emph{High Noon}, and Tiomkin's ballad "Do Not Forsake Me"
                which was another notable country song sung by Tex Ritter.
            \end{itemize}
        \end{itemize}  
    \end{description}
    \item \underline{Rock:}
    \begin{description}
        \item[Facts:]
        \item[] 
        \begin{itemize}
            \item Rock music or Rock and roll was a popular music type that had a specific style. Generally this type of music
            is heard with drummer's hard accents on beats two and four, and the inclusion of electronically amplified guitars (electric guitars).
            \item There was a clear association between rock and teenagers.
            \item Rock music became popular and king when Bill Haley's "Rock Around the Clock" became listed as the first rock tune to hit the
            Billboard's hits chart for 8 weeks. 
            \item Many notable figures of classic rock include Elvis Presley, Chuck Berry, Little Richard, and many more. 
        \end{itemize}
        \item[Notable Films with Songs:]
        \item[] 
        \begin{itemize}
            \item With this rise of rock music being listened, it was not far away from being incorporated into Hollywood films. The song
            "Rock Around the Clock" arose to popularity due to its appearance in the film \emph{The Blackboard Jungle} (1955) featuring juvenile delinquents.
            \item Hollywood took this new genre to benefit financially by making films that used this music to expand its audience. Other films include \emph{The
            Girl Can't Help It}, \emph{Love Me Tender} (1956), and \emph{Jailhouse Rock} (1957).
        \end{itemize}  
    \end{description}
\end{itemize}
\fi

\ifdraft{}
\section*{\underline{Paragraph 1: Introduction}}
\fi
In the 1950s arose numerous songs and a variety of popular music styles such 
as jass, country, and rock. Newer styles generally had hard accents, rapid 
notes and increased volume and intensity are some of the characteristics of popular
music styles in the 1950s. Generally newer styles of music incorporated in films began
to be associated with scenes of violence and sex.

\ifdraft{}
\section*{\underline{Paragraph 2: Jazz}}
\fi
Born from film noir, a dark and pessimistic film style, the sound of jazz featured a music type that utilizes cutting and coarse melodic line 
that is created generally from a saxophone or trumpet. Jazz was mainly employed to suggests seedier aspects of life such as sex and violence.
Futhermore, jazz was also employed for instrumental cues in scenes with a burst of emotion or rapid changes in mood. There were three notable films
that utilized jazz music: \emph{A Streetcar Named Desire}, \emph{Touch of Evil}, and \emph{Anatomy of a Murder}. In \emph{A Streecar Named Desire}, 
it was set in New Orleans which was the birthplace of jazz music, Alex North positions jazz source music of the film is naturally placed in all settings 
like barrooms and restaurants which entails an atmosphere of lust, instability, and passion. The underscoring nature of jazz was utilized in suggesting 
a sexually charged atmosphere of Stanley's House. There is a slow transition throughout the film from popular music to instrumental cues such as the rape of his 
sister-in-law. The film \emph{A Touch of Evil}, has two versions, was the last significant film noir that was surrounded by controversy on its production. The 
film incorporates many popular music styles in many of its scenes. Loud and intense jazz music is utilized during Hank Quinlan's murder of Uncle Joe Grande, which 
associates violence. Finally the film \emph{Anatomy of a Murder}, directed by Otto Preminger, jazz was mainly used during the contribution of discussions in regards 
to rape, semen, and "panties".


\ifdraft{}
\section*{\underline{Paragraph 3: Country}}
\fi
Country or western songs was another popular music type. Its usage in films was pioneered by Dimitri Tiomkin, a veteran Hollywood composer, 
that gave rise to theme songs. Theme songs became quickly popular as it served as a new source of income for recordings, and served as 
free publicity for the films that had the same name as the songs. A notable film that utilized this music type was \emph{High Noon} which was 
a classic film featuring a showdown between good and evil. The film lacked features of a true western of the age, however the true essence of 
the film is not the showdown, but the plot tensions that leads to the climax.
The film is more known for its scores such as the theme song \emph{High Noon}, and Tiomkin's ballad "Do Not Forsake Me" which was another notable 
country song sung by Tex Ritter.


\ifdraft{}
\section*{\underline{Paragraph 4: Rock}}
\fi
Rock music or Rock and roll was a popular music type that had a specific style. Generally this type of music
is heard with drummer's hard accents on beats two and four, and the inclusion of electronically amplified guitars (electric guitars).
There was a clear association between rock and teenagers. Rock music became popular and king when Bill Haley's "Rock Around the Clock" 
became listed as the first rock tune to hit the Billboard's hits chart for 8 weeks. Many notable figures of classic rock include Elvis 
Presley, Chuck Berry, Little Richard, and many more. With this rise of rock music being listened, it was not far away from being incorporated 
into Hollywood films. The song "Rock Around the Clock" arose to popularity due to its appearance in the film \emph{The Blackboard Jungle} (1955) 
featuring juvenile delinquents. Hollywood took this new genre to benefit financially by making films that used this music to expand its audience. 
Other films include \emph{The Girl Can't Help It}, \emph{Love Me Tender} (1956), and \emph{Jailhouse Rock} (1957) gave rise to many stars including Elvis Presley.


\end{document}