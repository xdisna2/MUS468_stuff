\documentclass[12pt]{article}
\usepackage[margin=1in]{geometry}
\usepackage[doublespacing]{setspace}
\usepackage{indentfirst}
\newif{\ifdraft}
\draftfalse

\begin{document}
Anthony Le

Prof. Roger Hickman

MUS468 (Section 2)

\today

\begin{center}
    \large Music in \emph{2001: A Space Odyssey}
\end{center}

\ifdraft{}
\section*{\underline{Prompt:}}
\noindent Choose one of the following films and describe how music contributes to the effect and quality of the film: 
Psycho, To Kill a Mockingbird, The Graduate, or 2001: A Space Odyssey.

\section*{\underline{Brainstorm:}}
\begin{itemize}
    \item 2001: A Space Odyssey
    \item \underline{List of Songs:}
    \begin{itemize}
        \item Richard Strauss: opening to Also sprach Zarathustra (1896)
        \item Johann Strauss Jr: Blue Danube Waltz (1867)
        \item Aram Khachaturian: Gayane Ballet Suite (942)
        \item Gyéray Ligeti: Atmospheres (1961) [orchestra, no voices]
        \item Gybrgy Ligeti: Lux Aeterna (1966) voices, no orchestra]
        \item Gyérgy Ligeti: Requiem (1965) [voices and orchestra]
        \item Gydrgy Ligeti: Aventures (1962)
    \end{itemize}
    \item Similar to leimotifs, Kubrick uses these borrowed music pieces to associate different
    events or scenes in the film. 
    \item For example, works from Richard Strauss are accompanying moments of evolution, Johann Strauss 
    and Khachaturian's works accomapny scenes of space travel. 
    \item Unlike how film music is utilized for certain details of the scenes such as character mood or movements,
    each of these songs are mainly detached from what is happening in the drama. 
    \item Music from Ligeti, which was used without permission, is utilized not for its melody and rhythm, but is used to emphasize 
    texture and sound masses a rather avant-garde style of the time.
    \item In the new era of music, Kubrick changes the meaning of how adapated works are incorporated into films. Rather than
    adapting or modifying previously made work to fit the needs of the film, these works can be left intact to allow for coexistance 
    with the visual elements of the film.
\end{itemize}
\fi

\ifdraft{}
\section*{\underline{Paragraph 1: Film Introduction}}
\fi
The film \emph{2001: A Space Odyssey} (1968), directed by Stanley Kubrick, expands upon Arthur C. Clarke's short story
"The Sentinel". Both individuals made a science fiction tale in regards to human evolution from ape to man to 
ultimately a higher spiritual being. This tale is derived as a science-fiction version about the idea of Christian theology
that suggests man will evolve into a higher form after the turn of the second milleunium.

\ifdraft{}
\section*{\underline{Paragraph 2: Adapted score}}
\fi
\emph{2001: A Space Odyssey} (1968) was well known not only for its visual effects, but also its use of adapted music.
Originally Kubrick, in collaboration with Alex North, was going to use the original score for \emph{2001}. However, Kubrick decided
to utilize borrowed music from previously made works. This film restablishes the traditions of adapted scores in an avant-garde way.

\ifdraft{}
\section*{\underline{Paragraph 3: Songs Utilized in the film}}
\fi
The film borrows music from a variety of composers of various sources. From Richard Strauss: opening to Also sprach Zarathustra (1896), 
Johann Strauss Jr: Blue Danube Waltz (1867), Aram Khachaturian: Gayane Ballet Suite (942), and numerous of Ligeti's works are 
the number of composers and their respective works that Kubrick decided to use for the film. Similar to leimotifs, Kubrick uses these 
borrowed music pieces to associate different events or scenes in the film. For example, works from Richard Strauss are accompanying 
moments of evolution, Johann Strauss and Khachaturian's works accomapny scenes of space travel. Unlike how film music is utilized for 
certain details of the scenes such as character mood or movements, each of these songs are mainly detached from what is happening in the drama. 
Music from Ligeti, which was used without permission, is utilized not for its melody and rhythm, but is used to emphasize texture and sound masses, 
a rather avant-garde style of the time.

\ifdraft{}
\section*{\underline{Paragraph 4: Final thoughts}}
\fi
In the new era of music, Kubrick changes the meaning of how adapated works are incorporated into films. Rather than
adapting or modifying previously made work to fit the needs of the film, these works can be left intact to allow for coexistance 
with the visual elements of the film.

\end{document}