\documentclass[12pt]{article}
\usepackage[margin=1in]{geometry}
\usepackage[doublespacing]{setspace}
\usepackage{indentfirst}
\newif{\ifdraft}
\draftfalse

\begin{document}
Anthony Le

Prof. Roger Hickman

MUS468 (Section 2)

\today

\begin{center}
    \large Film Music Types
\end{center}

\ifdraft{}
\section*{\underline{Prompt:}}
\noindent \emph{The Birth of a Nation} is the first full length film with an original score. 
Breil uses three types of music: adaptations, arrangements, and original. 
Describe the differences between them and give examples from the score. 
Be sure to include some leitmotifs with your discussion of original music. 

\section*{\underline{Brainstorm:}}
\begin{itemize}
    \item \emph{The Birth of a Nation} types of music:
    \begin{itemize}
        \item adaptations
        \item arrangements
        \item original
    \end{itemize}
    \item MAKE SURE TO INCLUDE SOME LEITMOTIFS
    \item \underline{Adaptations:}
    \begin{description}
        \item[Definition:] The borrowing of a substantial portion of an existing composition
        for use in a film score. The music should remain largely intact and recognizable,
        although it can be altered or adapted to suit the needs of the film. 

        Adapting works allowed for the less need to having to compose new music for each new 
        film, rehearsal times are reduced due to the familiarity of the classical work from performers.
        Furthermore, since classical works are well known to many, the worry about audience's familiarity and understanding
        of the music in the film is gone. Finally, for classical works there is very little copyright protections, 
        hence when deciding what music to use for these films, classical music was a good choice since there is no 
        convoluted protocal or rules to follow in terms of copyright laws.

        These songs are generally borrowed from works of the nineteenth-century composers since these works are generally
        not protected by copyright laws.
        \item[Example Music:] Carl Maria von (Weber's Der Freischutz), Franz von (Suppe's Light Cavalry) Overture, 
    \end{description}
    \item \underline{Arrangements:}
    \begin{description}
        \item[Definition:] A new setting of a previously composed melody. Unlike an adaptation,
        an arrangement borrows only a melody, which is then given a newly composed accompaniment.
        \item[Example Music:] "Where Did You Get That Hat?" (1888) a popular tune. Cameron seeks refuge in cabin scene utilizes Auld
        Lang Syne , Slaves dancing to "Turkey in the Straw" 
    \end{description}
    \item \underline{Original:}
    \begin{description}
        \item[Definition:] Music that is newly composed, there is no borrowing or adapting old works, 
        it is original music for the film.
        \item[Example Music:] "The Motif of Barbarism" (utilizes the syncopated rhythmic gestures found in music of
        African Americans), "The Perfect Song" (Love theme of Ben and Elsie)
    \end{description}
\end{itemize}
\fi

\ifdraft{}
\section*{\underline{Paragraph 1: Introduction}}
\fi
The film, \emph{The Birth of a Nation}, was the first masterpiece in the history of film that elevated filmaking
to become an art form that its known today. Other than being the first film to have an original score, the content 
of the film glorifies the Ku Klux Klan and ultimately represents the defense side to segregation in the South. The 
film directed by D. W. Griffith, utilizes three types of music: adaptations, arrangements, and original. The score 
for the film was composed by Joseph Carl Briel, and since the film had an original score, very few theaters showcased
the film since not only the producers were in charge of the film's music, but those were the only theaters that can support 
a full orchestra to play the music alongside the film. 

\ifdraft{}
\section*{\underline{Paragraph 2: Adaptations}}
\fi
The first type of music utilized in \emph{The Birth of a Nation} is adaptations. Music in films that are adaptations are substantially
borrowed from an existing composition to be used in the film. The music that is borrowed or adapted into the film is largely intact and recognizable
with some alterations to suit the needs of the film such as mood or tone of a scene. Generally music that is adapted comes from classical works.
The reason why classical works are more commonly adapted into films is because they allowed less of a need to compose new music for a film.
If a classical piece is able to convey such emotion or mood that the film needs, why not use it in the film? Furthermore, classical works are familiar 
to many from the performers to audience hence there is less worry about how the music affects the tone and emotion of a scene because many have
heard and understand its meaning. Finally, classical works have very little copyright protections, hence other than familiarity there is very little
difficulty or punishment for incorporating classical pieces into films. There are many adaptations are used in many scenes of the film such as 
Ludwig van Beethoven's Symphony No. 6 (\emph{The Storm}), and Richard Wagner's \emph{Rienzi} Overture, and \emph{Ride of the Valkyrie}. Both of Wagner's
pieces are adapted into the horse-riding rescue scene of the Ku Klux Klan. 


\ifdraft{}
\section*{\underline{Paragraph 3: Arrangements}}
\fi
The second type of music that was utilized in this film are arrangements. Similar to adaptations where the source of the song comes from a borrowing of
previous works, only the melody is utilized rather than large portions of a pre-existing work. This melody is then given a newly composed accompaniment
to be called an arrangement since it is an arrangement or both old and newly composed works rather than an adaptation where large portions of old works
are adapted into the film with minimal changes. Some music in \emph{The Birth of a Nation} that are arrangements are spread along in various scenes of the film. 
A popular tune of the time \emph{"Where Did You Get That Hat?"} was used to envoke a plafyfullness in scenes where Tod and Duke become playful, and when they die 
on the battlefield. \emph{"Auld Lang Syne"} plays when Cameron, a southerner, seeks refuge in a cabin with former Union soldiers suggesting acceptance and reconciliation.
Arrangements of melodies that are well-known to the populus are used primarily to arouse emotions and set the mood of a scene.

\ifdraft{}
\section*{\underline{Paragraph 4: Original}}
\fi
The final type of music that was utilized in the film \emph{The Birth of a Nation} was original music. Unlike the two previously mentioned music types: arrangements
and adaptations, original music is music that is newly composed, there is no borrowing or adapting previously made works, it is the original music for the film. Some
songs that are considered original for the film include \emph{"The Motif of Barbarism"} which utilizes the syncopated rhythmic gestures found in music of African Americans,
is utilized to represent Africans in general and the theme for mulatto Silas Lynch. Another song that was considered original is actually one of the six leitmotifs. This 
original song was named \emph{"The Perfect Song"}, and it represented the love theme for Elsie and Ben. This love theme became the best-known original song for the film.

\end{document}