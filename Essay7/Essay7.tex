\documentclass[12pt]{article}
\usepackage[margin=1in]{geometry}
\usepackage[doublespacing]{setspace}
\usepackage{indentfirst}
\newif{\ifdraft}
\draftfalse

\begin{document}
Anthony Le

Prof. Roger Hickman

MUS468 (Section 2)

\today

\begin{center}
    \large Music in Animations
\end{center}

\ifdraft{}
\section*{\underline{Prompt:}}
\noindent The 1990s saw a great variety of film music. Choose one of the following movie categories and 
describe the music and composers of specific film scores: history movies, animations, crime films, and dramas.

\section*{\underline{Brainstorm:}}
\begin{itemize}
    \item Film Scores to discuss:
    \begin{itemize}
        \item The Lion King
        \item Beauty and the Beast
        \item The Little Mermaid
    \end{itemize}
    \item Early 1980s, animated movies were not frequently produced like other film genres. However, with the rise of 
    \emph{The Little Mermaid} this film gave a revitalization to animated musicals. Disney studios being the backing force
    for the revival of animation returned to creating animations with certain characteristics of the past classics: a fairy story with 
    a strong female role, comic sidekicks for both antagonist and protagonist, and the incorporation of numerous engaging songs. 
    \item A leading figure behind Disney's resurgence is Alan Menken, who became one of the most honored figures in the history of the Academy.
    \item \underline{The Little Mermaid:}
    \begin{description}
        \item[Plot]
        \item[] 
        \begin{itemize}
            \item \emph{The Little Mermaid} retells an altered happier version of Hans Christian Anderson story.
            \item The film is set in an exotic underwater setting which allws Menken to incoporate non-Western sounds.
        \end{itemize}
        \item[Film Music Usage:]  
        \item[] 
        \begin{itemize}
            \item Menken weaves in a number of Caribbean musical elements alongside traditional styles. 
            \item For example, Sebastian the crab vibes to two calypso numbers \emph{Under the Sea}, and \emph{Kiss the Girl}.
            \item Having many songs, Menken ignores the cliches with previously animated cartoons such as nonstop music and "mickey mousing".
        \end{itemize}
    \end{description}
    \item \underline{Beauty and the Beast:}
    \begin{description}
        \item[Plot]
        \item[] 
        \begin{itemize}
            \item The animated film \emph{Beauty and the Beast} uses the same elements as mentioned before on a French fair tale in addition 
            to some adult themes. 
        \end{itemize}
        \item[Film Music Usage:]  
        \item[] 
        \begin{itemize}
            \item Just like in the little mermaid, Menken's musical choices are reflective on the locale. This 
            was evident in the song "Be Our Guest" with Lumiere leading kitchen objects in a spectacular sequence of dances and the French revue.
            \item Meken adds in some human features to inanimate kitchen objects to elevate the comedy, romance, and enchanted Disney magic
            that is standard in these films. 
        \end{itemize}
    \end{description}
    \item \underline{The Lion King:}
    \begin{description}
        \item[Plot]
        \item[] 
        \begin{itemize}
            \item Disney does it again by turning to a local African legend by producing \emph{The Lion King} (1994).
        \end{itemize}
        \item[Film Music Usage:]  
        \item[] 
        \begin{itemize}
            \item Unlike the previous films where the musical numbers are sung generally by the characters themselves,
            in this animated film the numbers are sung by unseen voices. For example, "The Circle of Life" is an underscore
            representing the birth of the main character Simba. However, if one were to watch the film, none of the animals
            are singing the song.
            \item Throughout the film there is a mixing of African choral sound with the Western popular music style.
            \item This mix gives a breath of freshness to the quality of the score.
        \end{itemize}
    \end{description}
\end{itemize}
\fi

\ifdraft{}
\section*{\underline{Paragraph 1: Introduction}}
\fi
Early 1980s, animated movies were not frequently produced like other film genres. However, with the rise of 
\emph{The Little Mermaid} this film gave a revitalization to animated musicals. Disney studios being the backing force
for the revival of animation returned to creating animations with certain characteristics of the past classics: a fairy story with 
a strong female role, comic sidekicks for both antagonist and protagonist, and the incorporation of numerous engaging songs. 


\ifdraft{}
\section*{\underline{Paragraph 2: The Little Mermaid}}
\fi
{The Little Mermaid} retells an altered happier version of Hans Christian Anderson story.
The film is set in an exotic underwater setting which allws Menken to incoporate non-Western sounds.
Menken weaves in a number of Caribbean musical elements alongside traditional styles. 
For example, Sebastian the crab vibes to two calypso numbers \emph{Under the Sea}, and \emph{Kiss the Girl}.
Having many songs, Menken ignores the cliches with previously animated cartoons such as nonstop music and "mickey mousing".


\ifdraft{}
\section*{\underline{Paragraph 3: Beauty and the Beast}}
\fi
The animated film \emph{Beauty and the Beast} uses the same elements as mentioned before on a French fair tale in addition 
to some adult themes.
Just like in the little mermaid, Menken's musical choices are reflective on the locale. This 
was evident in the song "Be Our Guest" with Lumiere leading kitchen objects in a spectacular sequence of dances and the French revue.
Meken adds in some human features to inanimate kitchen objects to elevate the comedy, romance, and enchanted Disney magic
that is standard in these films. 


\ifdraft{}
\section*{\underline{Paragraph 4: The Lion King}}
\fi
Disney does it again by turning to a local African legend by producing \emph{The Lion King} (1994).
Unlike the previous films where the musical numbers are sung generally by the characters themselves,
in this animated film the numbers are sung by unseen voices. For example, "The Circle of Life" is an underscore
representing the birth of the main character Simba. However, if one were to watch the film, none of the animals
are singing the song.
Throughout the film there is a mixing of African choral sound with the Western popular music style.
This mix gives a breath of freshness to the quality of the score.

\end{document}